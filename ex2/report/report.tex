\documentclass{article}

\usepackage[left=2.5cm, right=2.5cm, top=2.5cm, bottom=2.5cm]{geometry}
\usepackage{graphicx}
\usepackage{physics}
\usepackage{algpseudocode}
\usepackage{subcaption}
\usepackage{hyperref}


\title{Exercise 2, TFY4235 Computational physics}
\author{Martin a. Johnsrud}
\vspace{-8ex}
\date{}


\begin{document}
    \maketitle
    \section*{Introduction}
    This report documents the simluation of magnons, as described in~\cite{exercise}.

    \section*{Theory}
    \subsection*{Units}
    The Hamiltonian, as well as the equations of motion inc \cite{exercise} defines some natural units for the problem: 
    \begin{itemize}
        \item Energy: $[\mathcal H] = [J \hbar^2 s^2]$
        \item Magnetic field: $[\vec B] = [\mu \vec S ]$
        \item Anisotropy: $[d_z] = [J]$
        \item Time : $[t] = [\gamma J \hbar s]^{-1}$
    \end{itemize}
     where $s$ is the spin of the particles. ($s=1/2$ for electrons) This is included so that $|\vec S|\in[0, 1]$. The defining dimensionfull constants of the system is thus the spin $\hbar s$, the cupling $J$, the magnetic moment $\mu$ and the gyromacnetic ratio $\gamma$.
    \subsection*{Indices}
    For easy implementation, the Hamiltonian can be written on index form and using the units as described above
    \begin{equation*}
        \mathcal{H}(S; d_z, a, B) = -\frac{1}{2} J \sum_{\langle i, j \rangle, a} S_{i, a} S_{j, a} - d_z \sum_{j} (S_{j,3})^2 -  \sum_{j, a} B_{j, a} S_{j,a}.
    \end{equation*}
    Here, $J\in\{-1, 0, 1\}$, $i\in\{1, ..., N\}$ is the site index, $a$ is vector component index. The effective field can be written
    \begin{align*}
        H_{k, b} = - \pdv{\mathcal{H}}{S_{k, b}} = \frac{1}{2} J \sum_{\langle i, j \rangle, a} (S_{i, a}\delta_{j,k}\delta_{a, b} + S_{j, a}\delta_{i,k}\delta_{a, b}) + 2 d_z \sum_{j} S_{j,3} \delta_{b, 3}\delta_{j, k} +  \sum_{j, a} B_{j, a} \delta_{k,b},
    \end{align*}
    using the vector triple product identity $\vec A \times (\vec B \times \vec C) = (\vec A \cdot \vec B) \vec C - (\vec A \cdot \vec C) \vec B$. The first sum becomes
    \begin{equation*}
        \frac{1}{2}\sum_{\langle i, j \rangle, a} (S_{i, a}\delta_{j,k}\delta_{a, b} + S_{j, a}\delta_{i,k}\delta_{a, b}) = \frac{1}{2}\sum_{\langle i, j \rangle} (S_{i, b}\delta_{j,k} + S_{j, b}\delta_{i,k}) = \frac{1}{2}\sum_{\langle j, i \rangle} 2S_{i, b} \delta_{j, k} = \sum_{j \in \mathrm{NN}_k} S_{j, b},
    \end{equation*}
    where $\mathrm{NN}_k$ are the set of nearest negihbours of lattice point $k$. The Landua-Lifshitz-Gilbert equation for the time evolution of the system is then
    \begin{align}
        \dv{t} S_{j, a} &= - \frac{1}{(1 + \alpha^2)}\left[\sum_{b c}\varepsilon_{abc}S_{j, b}H_{j,c} + \alpha \sum_{b}\left(S_{j, b}S_{j, b}H_{j, a} - S_{j, b}H_{j, b}S_{j, a}\right)\right], \label{LLG} \\
        H_{k, b} &= J\sum_{j \in \mathrm{NN}_k} S_{j, b} + 2d_z S_{k, 3} \delta_{k, 3} +  B_{k, b}. \label{effective field}
    \end{align}
    
    \section*{Implementation}
    The main object of the simulation is a NumPy-array \verb|S| of shape \verb|(T, N, 3)|. This contains the components of each of the $N$ spins at each timestep. The function \verb|integrate| then runs a loop, calling the implementation of the Heun method \verb|heun_step|. The index notation laid out in the Theory section allows for straight forward implementation of the LLG equation using NumPy's \verb|einsum|-function. \verb|LLG| takes as arguments \verb|S| and the needed parameters. Then, it first evaluates the first sum of (\ref{LLG}) using two nested \verb|einsum|-functions, as well as an implementation of the Levi-Civita tensor. If $\alpha \neq 0$, it then evaluates the second sum. \verb|LLG| calls \verb|get_H|. This functions implements (\ref{effective field}), using NumPy's \verb|roll|-function to sum over all nearest negihbours. \verb|LLG| then returns \verb|dtS|, a NumPy-array containing the time derivative of $S$.


    \section*{Results}
    

    \bibliography{report}
    \bibliographystyle{plain}

\end{document}