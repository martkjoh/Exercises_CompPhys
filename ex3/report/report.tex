\documentclass{article}

\usepackage[left=2.5cm, right=2.5cm, top=2.5cm, bottom=2.5cm]{geometry}
\usepackage{graphicx}
\usepackage{physics}
\usepackage{subcaption}
\usepackage{hyperref}


\title{Exercise 3, TFY4235 Computational physics}
\author{Martin Johnsrud}
\vspace{-8ex}
\date{}


\begin{document}
    \maketitle
    \section*{Introduction}
        This is an implementation of \cite{exercise}.
    \section*{Theory}
    The diffusion equation, can be written as
    \begin{equation*}
        \Delta t \pdv{t} C(z, t) = \Delta t \left(K(z) \pdv[2]{z} + \dv{K(z)}{z}\pdv{z}\right) C(z, t) = \mathcal{D} C(z, t).
    \end{equation*}
    Discretizing the spatial part, and applying boundary conditions, gives
    \begin{equation*}
        \Delta t \pdv{t} C_n(t) = \mathcal{D}_{nm} C_n(t) + S_n(t),
    \end{equation*}
    where 
    \begin{equation*}
        \mathcal{D} =
        \begin{pmatrix}
            4\alpha K_0 + 2\Gamma & - 4\alpha K_0 & 0 & \dots&0 \\
            \alpha K_1'/2 - 2\alpha K_1 & 4 \alpha K_1 & - \alpha K_1'/2 - 2\alpha K_1 \\
            0 & \ddots & \ddots & \ddots & 0\\
            0 & \dots &\alpha K_{N-1}'/2 - 2\alpha K_{N-1} & 4 \alpha K_{N-1} & - \alpha K_{N-1}'/2 - 2\alpha K_{N-1} \\
             0 & \dots & 0 & 4\alpha K_N & 4\alpha K_N
        \end{pmatrix},\quad
        S(t) = 
        \begin{pmatrix}
            2\Gamma C_\mathrm{eq}(t) \\
            0\\
            \vdots \\
            0
        \end{pmatrix}
    \end{equation*}
    The Chranck-Nichelson scheme then yields
    \begin{equation*}
        C_n^{i+1}  = C_n^i + \frac{1}{2} (\mathcal{D}_{nm} C_m^i + S_n^i) + \frac{1}{2} (\mathcal{D}_{nm} C_m^{i+1} + S_n^{i+1}),
    \end{equation*}
    so 
    \begin{equation*}
        \left(\delta_{nm} - \frac{1}{2} \mathcal{D}_{nm}\right) C_{m}^{i+1} = \left(\delta_{nm} + \frac{1}{2} \mathcal{D}_{nm}\right) C_m^i + \frac{1}{2}(S_n^i + S_n^{i+1})
    \end{equation*}
    \bibliography{report}
    \bibliographystyle{plain}

\end{document}